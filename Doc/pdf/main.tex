% SRS_Application_echange.tex
% Généré par assistant — contient le SRS en français et deux diagrammes TikZ intégrés :
% - diagramme entité-relation (BDD)
% - diagramme de classes (serveur)
% Compilation : pdflatex -interaction=nonstopmode ./pdf/main.tex

\documentclass[11pt,a4paper]{article}
\usepackage[T1]{fontenc}
\usepackage[utf8]{inputenc}
\usepackage[french]{babel}
\usepackage{lmodern}
\usepackage{geometry}
\usepackage{hyperref}
\usepackage{graphicx}
\usepackage{enumitem}
\usepackage{longtable}
\usepackage{caption}
\usepackage[inkscapeformat=png]{svg}
\geometry{margin=2.5cm}
\hypersetup{pdftitle={SRS - Application d\'echange local},colorlinks=true,linkcolor=blue}

\begin{document}

\begin{center}
  {\LARGE \textbf{Software Requirements Specification (SRS)}}\\[8pt]
  {\large \textbf{Application d\'echange local de biens et de comp\'etences}}\\[6pt]
  Version : 1.0 \\ Auteur : Simon \\ Date : ---
\end{center}

\vspace{6pt}\hrule\vspace{10pt}

\section*{1. Introduction}
\subsection*{1.1 Objectif du document}
Décrire les exigences fonctionnelles et non fonctionnelles pour le serveur Java
monolithique, implémenté avec Jetty et Hibernate ORM, compatible SQLite et
PostgreSQL, en appliquant les principes SOLID.

\subsection*{1.2 Portée du système}
Serveur backend monolithique : API REST pour le frontend, logique métier,
persistance via Hibernate, notifications, messagerie, gestion des échanges,
syst\`eme d\'avis et d\'authentification.

\subsection*{1.3 D\'efinitions, acronymes et abbr\'eviations}
\begin{itemize}
  \item API : Application Programming Interface
  \item SGBD : Syst\`eme de gestion de base de donn\'ees
  \item SOLID : principes de conception objet
  \item JWT : JSON Web Token
  \item ORM : Object Relational Mapping
  \item ACID : Atomicit\'e, Coh\'erence, Isolation et Durabilit\'e
\end{itemize}

\subsection*{1.4 R\'ef\'erences}
Projet de synth\`ese : Application d\'echange local de biens et de
comp\'etences.

\section*{2. Description g\'en\'erale}
\subsection*{2.1 Perspective du produit}
Monolithe Java, ex\'ecution sur JVM 17+, serveur embarqu\'e Jetty, persistance
via Hibernate ORM. L\'application expose une API REST et persiste les donn\'ees
dans une base relationnelle (PostgreSQL en production, SQLite possible pour
d\'emo et/ou tests).

\subsection*{2.2 Fonctions du produit}
\begin{itemize}
  \item Authentification (JWT), gestion des profils.
  \item CRUD annonces (objets, comp\'etences).
  \item Gestion des demandes d\'echange et messagerie.
  \item Avis et notation.
  \item Notifications (in-app \_+ email\_).
\end{itemize}

\subsection*{2.3 Caract\'eristiques des utilisateurs}
\begin{itemize}
  \item Visiteur (lecture limit\'ee).
  \item Utilisateur authentifi\'e (cr\'eation/modification d\'annonces, \'echanges).
  \item Administrateur (gestion, suppression abusive).
\end{itemize}

\subsection*{2.4 Contraintes g\'en\'erales}
\begin{itemize}
  \item Application monolithique.
  \item Persistance via Hibernate.
  \item Serveur embarqu\'e Jetty.
  \item Compatibilit\'e : SQLite (local et/ou dev) et PostgreSQL (pour d\'eploiement).
  \item Respect minimal de la confidentialit\'e (pas d' adresse/localisation exacte
        expos\'ee).
\end{itemize}

\subsection*{2.5 Hypoth\`eses et d\'ependances}
\begin{itemize}
  \item Frontend respecte conventions des API REST.
  \item Serveur SMTP disponible pour envois d\'e-mails (non-impl\'ement\'e).
  \item Environnements Java 17+ disponibles.
\end{itemize}

\section*{3. Exigences fonctionnelles}
\subsection*{3.1 Authentification}
\begin{itemize}
  \item RF-AUTH-001 : Inscription (email, nom, localisation approximative).
  \item RF-AUTH-002 : Connexion et \'emission d'un JWT.
  \item RF-AUTH-003 : Stockage des mots de passe hach\'es (BCrypt).
\end{itemize}

\subsection*{3.2 Gestion des utilisateurs}
\begin{itemize}
  \item RF-USR-001 : \'Edition du profil (bio, photo).
  \item RF-USR-002 : Calcul et retour de la note moyenne.
\end{itemize}

\subsection*{3.3 Annonces (objets)}
\begin{itemize}
  \item RF-OBJ-001 : Cr\'eation d'annonce (titre, description, cat\'egorie, images,
        disponibilit\'e).
  \item RF-OBJ-002 : Recherche/filtrage par mots-cl\'es/cat\'egorie.
  \item RF-OBJ-004 : Consultation d'une annonce.
\end{itemize}

\subsection*{3.4 Annonces (comp\'etences)}
\begin{itemize}
  \item RF-COMP-001 : Cr\'eation, recherche, consultation d'annonces de comp\'etences.
\end{itemize}

\subsection*{3.5 \'Echanges}
\begin{itemize}
  \item RF-ECH-001 : Envoi de demande d'\'echange.
  \item RF-ECH-002 : Acceptation/refus par le propri\'etaire.
  \item RF-ECH-004 : Historique des \'echanges.
\end{itemize}

\subsection*{3.6 Messagerie}
\begin{itemize}
  \item RF-MSG-001 : Cr\'eation de conversation \`a l'acceptation.
  \item RF-MSG-002 : Envoi et persistance de messages.
\end{itemize}

\subsection*{3.7 Avis}
\begin{itemize}
  \item RF-AVIS-001 : Notation 1--5 \"etoiles\".
  \item RF-AVIS-002 : Commentaire public.
\end{itemize}

\subsection*{3.8 Notifications}
\begin{itemize}
  \item RF-NOTIF-001 : Notification sur nouvelle demande.
  \item RF-NOTIF-002 : Notification de message.
  \item RF-NOTIF-003 : Envoi d'e-mail (si configur\'e).
\end{itemize}

\section*{4. Exigences non fonctionnelles}
\subsection*{4.1 Performance}
Temps de r\'eponse vis\'e : $<$ 300 ms (op\'erations standards en conditions
normales).

\subsection*{4.2 S\'ecurit\'e}
\begin{itemize}
  \item JWT pour authentification.
  \item Password4j pour mots de passe.
  \item Validation c\^ot\'e serveur.
  \item HTTPS obligatoire en production.
\end{itemize}

\subsection*{4.3 Fiabilit\'e}
Transactions ACID gr\^ace \`a JPA/Hibernate pour les op\'erations critiques.

\subsection*{4.4 Maintenabilit\'e (SOLID)}
L'architecture applicative doit suivre SOLID : SRP, OCP, LSP, ISP, DIP.

\subsection*{4.5 Scalabilit\'e}
Couches claires, contracts API, repository faiblement coupl\'es pour un futur
d\'ecoupage.

\subsection*{4.6 Compatibilit\'e}
API JSON, UTF-8. ORM Hibernate configur\'e pour PostgreSQL et SQLite (dialects
configurables).

\section*{5. Interfaces externes}
\subsection*{5.1 API REST}
Liste complète des routes exposées :
\begin{itemize}
  \item GET /test
  \item GET /users
  \item GET /users/{id}
  \item POST /users
  \item PUT /users/{id}
  \item DELETE /users/{id}

        % Auth routes commentées
        %\item POST /auth/login
        %\item POST /auth/logout

  \item GET /adverts
  \item GET /adverts/{id}
  \item POST /adverts
  \item PUT /adverts/{id}
  \item DELETE /adverts/{id}
  \item GET /adverts/search

  \item GET /applications
  \item GET /applications/{id}
  \item POST /applications
  \item POST /applications/{id}/accept
  \item POST /applications/{id}/reject
  \item DELETE /applications/{id}

  \item GET /messages
  \item GET /messages/{id}
  \item POST /messages
  \item GET /messages/conversation/{userId}

  \item GET /notifications
  \item GET /notifications/{id}
  \item POST /notifications/{id}/read
  \item POST /notifications

        % Reviews/commentaires commentés
        %\item POST /reviews
        %\item GET /reviews/{userId}
\end{itemize}

\subsection*{5.2 Base de donn\'ees}
Tables principales : utilisateur, annonce, echange, message, avis,
notification.

\begin{figure}[!htbp]
  \centering
  \includegraphics[width=0.9\textwidth,height=0.9\textheight,keepaspectratio]{assets/entityrelationdiagDB.png}
  \caption{Schéma entité relation de la base de données}
  \label{fig:mon_label_er}
\end{figure}

\subsection*{5.3 Services externes}
\begin{itemize}
  \item SMTP (envoi d'e-mails) (Non impl\'ement\'e).
  \item (Optionnel) service d'h\'ebergement d'objets multim\'edias (Non impl\'ement\'e).
\end{itemize}

\section*{6. Contraintes techniques}
\subsection*{6.1 Environnement d'ex\'ecution}
\begin{itemize}
  \item Java 17+
  \item Javalin
  \item Maven
\end{itemize}

\subsection*{6.2 D\'ependances logicielles}

\begin{itemize}
  \item Javalin
  \item Hibernate ORM
  \item PostgreSQL ou SQLite
  \item JUnit
  \item Password4j
\end{itemize}

\subsection*{6.3 Configuration Hibernate}
Par d\'efaut : PostgreSQL. Option \`a passer : --database sqlite pour les tests
locaux.

\section*{7. Crit\`eres d'acceptation}
Fonctionnel :
\begin{itemize}
  \item flux d'\'echange complet
  \item authentification
  \item autorisations v\'erifi\'ees.
\end{itemize}

Non fonctionnel :
\begin{itemize}
  \item tests CI avec SQLite et/ou pr\'eprod PostgreSQL
  \item temps de r\'eponse conforme
  \item mots de passe hach\'es.
\end{itemize}

\section*{8. Annexes}
\subsection*{8.1 Diagrammes d'architecture}

\begin{figure}[!htbp]
  \centering
  \includegraphics[width=0.95\textwidth,height=0.95\textheight,keepaspectratio]{assets/classdiagserverTB.png}
  \caption{Diagramme de classe}
  \label{fig:mon_label_class}
\end{figure}

\end{document}
